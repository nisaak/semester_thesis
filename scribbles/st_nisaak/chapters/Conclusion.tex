% !TeX spellcheck = en_US
% !TeX encoding = UTF-8
% !TeX root = ../report.tex

\chapter{Conclusion}
\label{chp:Conclusion}

The evaluation of the proposed V-Disparity method for free-space detection shows the need for an auxiliary sensor, in our case a range finding lidar sensor. The stereocamera on it's own does not provide the desired quality of the road mask. To complete the task of creating a useful road mask, a second method was implemented, which projects scanned lidar points into the camera frame. This was done by first externally calibrating the two sensors, after meticulously measuring the relative pose and orientation. This approach was then compared to a automatic calibration method, which is available online.

\subsection{Future Work}
For future work, the disparity map generation should be improved. The quality of disparity maps generated by recent neural networks seem promising, such as PSMNet by Jlia et al.  \cite{DBLP:journals/corr/abs-1803-08669}. Resolving the bottleneck of the disparity map can shed light on the performance of the actual V-Disparity method. Keeping the initial ground mask undersegmented, and refining it by using it as seeds for region growing algorithms, may improve results.
Augmenting the code to be more computationally efficient should be considered as well. Using for loops in Python increases computing time, replacing loops with other functionalities can help.
