% !TeX document-id = {489a4b48-3347-4d5e-985a-4bd4e83eb476}
% !TeX spellcheck = en_US
% !TeX encoding = UTF-8
% !TeX TXS-program:compile = txs:///latexmk/[-pdf -silent -shell-escape -latexoption="-synctex=1" -output-directory="build" -r "docstyle/nomenclature_latexmkrc"]
% !TeX TXS-program:quick = txs:///compile | txs:///view

\documentclass[english,sp]{docstyle/IDSCreport}

\title{Free Space Segmentation for Gokart Application}
\subtitle{V-Disparity}
\author{Noah Isaak}
\ethid{13-929-476}
\semester{FS 2019}
\email{nisaak@student.ethz.ch}
\supervision{First Supervisor\\ Prof. Dr. Second Supervisor}
\identification{IDSC-XX-YY-ZZ}
\date{December 2018}
\keywords{Ground detection, V-Disparity}
\bibliography{bibliography}

\begin{abstract}
Road and ground detection are closely related key tasks for an autonomous ground vehicle. These computations should be robust and preferably be performed in real-time. This paper aims to show the implementation of the V-Disparity method for ground detection. The approach is based on classic computer vision and does not incorporate learning methods. Basis of the method is a disparity map, for which a row-wise histogram is computed. This V-disparity histogram robustly preserves geometric scene features and can be used for various tasks. Experimental results however show the shortcomings of the implementation and how they could be overcome. In the following, a second pipeline is introduced. A mapping from a Velodyne Vld-16 Lidar to a camera is computed. The binary obstacle - ground mask which is computed from the lidar's point cloud can then be projected onto the camera image. These labels can then be used for e.g. machine learning tasks.


\end{abstract}

\begin{document}
% !TeX spellcheck = en_US
% !TeX encoding = UTF-8
% !TeX root = ../report.tex

\chapter{Introduction}
\label{chp:Introduction}

The tasks of segmenting a scene into ground, obstacles and other labels is a well-researched topic. It is a fundamental part of any autonomous ground vehicle, and lays the basis for many different tasks, such as planning, safety features or scene understanding. Over the last few years, solutions to this task, which are based on machine learning methods, have deemed themselves to be robust and accurate. Their ability to generalise and be applied to a variety of scenes make them a reliable choice. There however, exist many approaches to the problem, which are based on classic computer vision, with implementations dating few decades back. \newline
The goal of this semester project is the implementation of a robust and accurate free space detection pipeline based on classic computer vision. The results of this pipeline can be used to e.g. label data for machine learning. If the pipeline does not perform as expected, a second pipeline is introduced, where the focus entirely lies on labeling data for machine learning tasks. 

This semester project is part of a more extensive research into autonomous driving, based a self-driving Gokart. The Gokart is basis for a variety of research topics, such as planning, control systems and sensor fusion. The testing environment is a modular indoor Gokart track. The ground is flat, with no uphill or downhill sections. The track is not affected by weather conditions. 
\newline


\section{Related Work}
Related work on this topic is extensive. After limiting my research to the method of V-Disparity, early implementations can be dated back to 2002, with \cite{V-disparity_nonflat} providing a robust and accurate method for road detection, even for non-flat geometries. A study on the U-V Disparity method can be found in \cite{Hu2005}, providing a real-time implementation of the stereovision based scene analysis. The method was improved and adapted since it's introduction, becoming more robust and computationally efficient, as shown in \cite{Kakegawa2018}. Aside from free space detection, U-V-Disparity can be used solely for obstacle and pedestrian detection, with U-V-Disparity acting as the underlying basis for a SVM Classifier, where the extraced ROIs are used for training. Iloie et al. implemented this framework in \cite{Iloie2014}.
% !TeX spellcheck = en_US
% !TeX encoding = UTF-8
% !TeX root = ../report.tex

\chapter{Method}
\label{chp:Method}

\section{V-Disparity}

This chapter explains the theoretical basis of the V-disparity method.

\section{Lidar Camera Projection}

More stuff
% !TeX spellcheck = en_US
% !TeX encoding = UTF-8
% !TeX root = ../report.tex

\chapter{Implementation}
\label{chp:Implementation}


\section{V-Disparity Implementation}
\label{vdisp_impl}


\section{Hardware}

The Gokart is fitted with a ZED Stereolabs camera, which was the basis for the V-Disparity method. The camera comes with a ZED SDK, which provides many functionalities, such as disparity map generation and point cloud computation.
In addition, a Velodyne VLD-16 Lidar is mounted on top of the Gokart. It offers 16 scanning lines with high accuracy.
More infos on the used hardware can be found on the manufacturer's website.

\section{Code implementation}

The method was implemented in the Python framework, while making use of the popular OpenCV library. OpenCV was used, mainly because of the existing methods already implemented, such as the probabilistic houghline transform, which was used for line-fitting. The OpenCV aims at real-time computer vision programming functions, which is a crucial property for autonomous vehicles.
\newline

Given the ZED Camera and the ZED SDK, data can be streamed from the Camera with few lines of code. Depending on the application, the resolution, framerate and other parameters can be adjusted accordingly. For disparity computation, range and quality can be modified, to fit the requirements.
The disparity map is computed as float32. For visualisation purposes, a normalisation and conversion to uint8 type needs to be done. 
For higher precision, the float32 disparity map can be used for further processing. 
To ease up on computation, the uint8 disparity map was used for the V-Disparity method. The number of histogram bins for every row histogram is then set to 255, one for every disparity value present in the image. OpenCV's calcHist function was used.


In order to avoid speckles and disjointed mask segments, all pixels are labeled according to their connectedness, using OpenCV's connectedComponents function. Only the largest connected region is kept, all other labels are discarded. 
% !TeX spellcheck = en_US
% !TeX encoding = UTF-8
% !TeX root = ../report.tex

\chapter{Results}
\label{chp:Results}

Results have been visually evaluated, as there is no ground truth available for this method. 
The evaluation have shown, that for this project's implementation, the V-Disparity method does not provide the desired results. The generated mask lacks temporal coherency, and for mid-range distances (>4m), important features are not detected, such as various obstacles or road boundaries. 
\newline
The reason for this can be found in the generated disparity map. Without post processing, the backprojection of the road requires an accurate disparity map. However the quality of the disparity map diminishes for further distances, edges are not well preserved and artifacts in low-textured areas lead to inaccurate masks.
\newline

\appendix
% !TeX spellcheck = en_US
% !TeX encoding = UTF-8
% !TeX root = ../report.tex

\chapter{AppendixChapter}
\label{chp:AppendixChapter}


\end{document}
